\documentclass[a4,12pt]{article}
\usepackage[utf8]{inputenc}
\usepackage[spanish]{babel}
\usepackage[margin=3cm]{geometry}
\usepackage{graphicx}
\usepackage{import}
\usepackage{color}
\usepackage{times}

\renewcommand{\familydefault}{\sfdefault}


\usepackage{hyperref}

\hypersetup{
    pdfborder = {0 0 0}
}

\title{Cálculo del tiempo del algoritmo de ordenación QuickSort para diferentes tamaños de problema}

\author{Lydia Martínez-Fortún Martínez}



\begin{document}

\maketitle



\begin{abstract}
En este documento en \LaTeX{} se explica el calculo del algoritmo de ordenación QuickSort hecho en octave y los tiempos de ejecución para diferentes tamaños de problemas.
\end{abstract}

\tableofcontents

\newpage

\section{Introducción al algoritmo}

En este documento se muestra como hacer un algoritmo de ordenación en octave. En este caso he implementado el algoritmo QuickSort, que lo que hace es elegir un pivote y dividiendo el vector en los mayores, menores o iguales a él.



\section{QuickSort}

%Explicar el pseudocodigo

\subsection{Pseudocódigo}

\subsection{Código en octave}



\section{Gráfica de tiempos de ejecución}


\bibliographystyle{plain}
\bibliography{referencias}

\end{document}

